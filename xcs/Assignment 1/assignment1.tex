
%%%%start parameter to change
\def\assignmentnum{1}
\def\assignmenttitle{XCS251 Assignment \assignmentnum : Key components of Bitcoin blockchain design}
\def\month{September}
\def\weekday{Sunday}
\def\day{11}
\def\time{11:59pm}
\def\postquestion{slack channel}
\def\submissiononline{Gradescope}
\def\latextemplate{assigment1-template.tex}

%%%%end parameter to change

\def\due{Due \weekday, \month\ \day\ at \time\ PT.}


\documentclass[11pt]{article}
\usepackage{epsfig,amsfonts,amsmath,amssymb,graphics,fullpage,xcolor}
\usepackage[colorlinks=true,linkcolor=blue,urlcolor=blue, citecolor=blue, hidelinks]{hyperref}
\hypersetup{colorlinks=true,citecolor=blue,linkcolor=blue,linktocpage=true}
\DeclareUrlCommand\url{\color{blue}}
\newcommand{\xor}{\oplus}
\newcommand{\deq}{\mathrel{:=}}

%%%%%%%%
\usepackage{enumitem}
\newenvironment{problems}
{\begin{enumerate}[label=\bfseries Problem \arabic*.,align=left,leftmargin=1em,labelwidth=1.5em]}
{\end{enumerate}}

\newenvironment{subparts}
{\begin{enumerate}[label=\bfseries \alph*.,align=right,leftmargin=1.5em]}
{\end{enumerate}}
%%%%%%%%
\usepackage[headsep=1cm,headheight=2cm]{geometry}
\usepackage{fancyhdr}
\pagestyle{fancy}
%%
%%
\lhead{\assignmenttitle}
\rhead{\thepage}
\cfoot{}
%%
\renewcommand{\headrulewidth}{0pt}
\renewcommand{\footrulewidth}{0pt}
\newcommand{\ruleskip}{\bigskip\hrule\bigskip}
\newcommand{\latex}{\LaTeX\xspace}
\newcommand{\Hint}{\noindent {\bf Hint: }}
\setlength\parindent{0pt}
\begin{document}

{\Huge\assignmenttitle}

\ruleskip


{\bf \due}

\vspace{0.2cm}
{\bf Guidelines}
\begin{enumerate}[itemsep=2pt]
    \item If you have a question about this assignment, we encourage you to post your question on \postquestion.
    \item Familiarize yourself with the collaboration and honor code policy before starting work.
\end{enumerate}

\vspace{0.2cm}

{\bf Submission Instructions} 
\vspace{0.2cm}

You should submit a PDF with your solutions online in \submissiononline. As long as the PDF is legible and
organized, the course staff has no preference between a handwritten and a typeset \LaTeX{} submission. If you wish
to typeset your submission and are new to \LaTeX{}, you can get started with the following:

\begin{itemize}
    \item Download and install \href{https://www.tug.org/texlive/}{Tex Live} or try \href{https://www.overleaf.com/}{Overleaf}.
    \item Submit the compiled PDF.
\end{itemize}

\vspace{0.2cm}

{\bf Honor code} 
\vspace{0.2cm}

We strongly encourage students to form study groups. Students may discuss and
work on assignment in groups. However, each student must write down the
solutions independently, and without referring to written notes from the joint
session. In other words, each student must understand the solution well enough
in order to reconstruct it by him/herself. In addition, each student should
write on the assignment the set of people with whom s/he collaborated. Further,
because we occasionally reuse problem set questions from previous years, we
expect students not to copy, refer to, or look at the solutions in preparing
their answers. It is an honor code violation to intentionally refer to a
previous year's solutions.  More information regarding the Stanford honor code
can be found at \href{https://communitystandards.stanford.edu/policies-and-guidance/honor-code}{https://communitystandards.stanford.edu/policies-and-guidance/honor-code}.

\vspace{0.2cm}
\newpage
\sloppy
{\bf Introduction} 
\vspace{0.2cm}

In this assignment, there are several questions related to key components of Bitcoin blockchain design. In particular, the first two problems will help you to understand why proof of work hash function needs to be designed carefully and why a binary Merkle tree is good enough to commit a list of elements. The third and fourth question will help you to become familiar with Bitcoin script, ScriptPubKey, ScriptSig, and multisig transactions. The final question examines the minimum proof size required for a lightweight client. By the end of this assignment, you should have a good understanding of proof of work hash function, Merkle proof, Bitcoin script and size of Bitcoin blockchain.


\vspace{0.2cm}

\begin{problems}


%%%%%%%%%%%%%%%%%%%%%%%%%%%%%%%%%%%%%%%%%%%%%%%%%%%%%%%%%%%%%%%%%%%%%%%%%%

\item 
{\bf A broken proof of work hash function.}
In class we discussed using a hash function 
$H:X \times Y \to \{0,1,\ldots,2^n-1\}$
for a proof of work scheme.
Once an $x \in X$ and a difficulty level~$D$ are published,
it should take an expected $D$ evaluations 
of the hash function to find a $y \in Y$ such that $H(x,y)<2^n/D$. 
Suppose that $X = Y = \{0,1\}^m$ for some $m$ (say $m=512$),
and consider the hash function 
\[    H:X \times Y \to \{0,1,\ldots,2^{256}-1\}  \qquad 
         \text{defined as $ H(x,y) \deq \text{SHA256}(x \xor y)$}.  
\]
Here $\xor$ denotes a bit-wise xor.
Show that this $H$ is insecure as a proof of work hash.
In particular, suppose $D$ is fixed ahead of time.
Show that a clever attacker can find a solution $y \in Y$ 
with minimal effort once $x \in X$ is published. 

{\bf Hint:} the attacker will do most of the work before $x$ is published.



%%%%%%%%%%%%%%%%%%%%%%%%%%%%%%%%%%%%%%%%%%%%%%%%%%%%%%%%%%%%%%%%%%%%%%%%%%

\item {\bf Beyond binary Merkle trees.}  Alice can use a binary Merkle
  tree to commit to a list of elements $S = (T_1, \ldots, T_n)$ so
  that later she can prove to Bob that $S[i] = T_i$ using an
  inclusion proof containing at most $\lceil \log_2 n \rceil$ hash
  values.  The binding commitment to $S$ is a single hash value. In
  this question your goal is to explain how to do the same using a
  $k$-ary tree, that is, where every non-leaf node has up to $k$
  children. The hash value for every non-leaf node is computed as the
  hash of the concatenation of the values of all its children.
\begin{subparts}
\item 
Suppose $S = (T_1, \ldots, T_9)$. Explain how Alice computes a
commitment to $S$ using a ternary Merkle tree (i.e., $k=3$).  
How does Alice later prove to Bob that $T_4$ is in $S$?
What values are provided in the proof?

\item
Suppose $S$ contains $n$ elements. What is the length of the proof, i.e., the number of values in the proof,
that proves that $S[i] = T_i$, as a function of $n$ and $k$? For example, if the proof is $\{x1, y2, y3, y4\}$ for some Merkle Tree, then the length of the proof is 4. 

\item 
For large $n$, if we want to minimize the length of the proof, is it better
to use a binary or a ternary tree?  Why?

\end{subparts}

%%%%%%%%%%%%%%%%%%%%%%%%%%%%%%%%%%%%%%%%%%%%%%%%%%%%%%%%%%%%%%%%%%%%%%%%%%
\newpage
\item {\bf Bitcoin script.}  Alice is on a backpacking trip and is
  worried about her devices containing private keys getting stolen. 
  She wants to store her bitcoins in such a way that they can be
  redeemed via knowledge of a password $P$. Accordingly, she stores
  them in the following {\tt ScriptPubKey} address:
\begin{center}
%\begin{minipage}{3in}
\begin{verbatim}
OP_SHA256
<0xeb271cbcc2340d0b0e6212903e29f22e578ff69bab5690ff654bcd123ae890aa>
OP_EQUAL
\end{verbatim}
%\end{minipage}
\end{center}


\begin{subparts}
\item
Write a {\tt ScriptSig} script that will successfully redeem this
UTXO given the password $P$. \\
{\bf Hint:} it should only be one line long.

\item
Suppose Alice chooses a six character password $P$.
Explain why her bitcoins can be stolen soon after her UTXO is posted
to the blockchain. 
You may assume that computing SHA256 of all six character passwords
can be done in reasonable time.  


\item
Suppose Alice chooses a strong 30 character passphrase $P$.  
Is the {\tt ScriptPubKey} above a secure way to protect her bitcoins?
Why or why not?  \\
{\bf Hint:} reason through what happens when she tries to redeem her bitcoins.


\end{subparts}

%%%%%%%%%%%%%%%%%%%%%%%%%%%%%%%%%%%%%%%%%%%%%%%%%%%%%%%%%%%%%%%%%%%%%%%%%%

\item {\bf BitcoinLotto:} 
  Suppose the nation of Bitcoinia decides to convert its national
  lottery to use Bitcoin. A trusted scratch-off ticket printing
  factory exists and will not keep records of any values
  printed. Bitcoinia proposes a simple design: 
  a weekly public address is published that holds the jackpot.
  This allows everyone to verify that the jackpot exists, namely
  there is a UTXO bound to that address that holds the jackpot amount. 
  Then a weekly run of tickets is printed
  so that the winning lottery ticket contains the correct private key 
  under the scratch material.
\begin{subparts}
\item If the winner finds the ticket on Monday and immediately claims
  the jackpot, this will be bad for sales because players will all
  realize the lottery has been won. Modify the design to use
  one of the locktime opcodes to ensure that the prize can only
  be claimed at the end of the week. 
  (of course, you cannot prevent the winner from proving ownership of
  the correct private key outside of Bitcoin).  
  The Bitcoin script opcodes are listed here: 
  \url{https://en.bitcoin.it/wiki/Script}.
  Alternatively, explain how to use a 2-out-of-2 multisig for this.

\item Some tickets inevitably get lost or destroyed.  Modify the
  design to roll forward so that any unclaimed jackpot from Week $n$ 
  can be claimed by the winner in Week $n+1$. 
  If the Week $n+1$ jackpot is unclaimed, then
  the jackpot from both weeks $n$ and $n+1$ can be claimed by the 
  winner of Week $n+2$, and so on.  Can you propose a design that
  works without introducing new ways for the lottery administrators to
  embezzle funds beyond what is possible on the basic scheme
  described at the beginning of the question?  
  You may assume that the lottery system runs for 1000 weeks, and then shuts
  down permanently. 

 {\bf Hint:} use multisig.

\end{subparts}


%%%%%%%%%%%%%%%%%%%%%%%%%%%%%%%%%%%%%%%%%%%%%%%%%%%%%%%%%%%%%%%%%%%%%%%%%%
\vspace{5mm}

\item {\bf Lightweight clients:} Suppose Bob runs an ultra lightweight
  Bitcoin client which receives the current head of the blockchain from a
  trusted source. This client has very limited memory and so it only
  stores the block header of the most recent block
  in the chain (the head of the chain), deleting any previous block headers.
\begin{subparts}
\item If Alice wants to send a payment to Bob, what information should
  she include to prove that her payment to Bob has been included in
  the block chain?

\item Assume Alice’s payment was included in a block $k$ blocks before
  the current head and there are exactly $n$ transactions per block. (For example, suppose that the head is block number $12$, and $k=12$, then Alice's payment is at block number $0$.) Estimate
  how many bytes this proof will require in terms of $n$ and $k$.
  Compute the proof size for $k=6$ and $n=1024$.

\end{subparts}

\end{problems}
\end{document}
